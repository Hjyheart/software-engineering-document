% XeLaTeX can use any Mac OS X font. See the setromanfont command below.
% Input to XeLaTeX is full Unicode, so Unicode characters can be typed directly into the source.

% The next lines tell TeXShop to typeset with xelatex, and to open and save the source with Unicode encoding.

%!TEX TS-program = xelatex
%!TEX encoding = UTF-8 Unicode

\documentclass[UTF8]{ctexart}
\usepackage{geometry}                % See geometry.pdf to learn the layout options. There are lots.
\geometry{letterpaper}                   % ... or a4paper or a5paper or ... 
%\geometry{landscape}                % Activate for for rotated page geometry
%\usepackage[parfill]{parskip}    % Activate to begin paragraphs with an empty line rather than an indent
\usepackage{graphicx}
\usepackage{amssymb}
\usepackage{enumerate}

% Will Robertson's fontspec.sty can be used to simplify font choices.
% To experiment, open /Applications/Font Book to examine the fonts provided on Mac OS X,
% and change "Hoefler Text" to any of these choices.

\usepackage{fontspec,xltxtra,xunicode}
\defaultfontfeatures{Mapping=tex-text}
\setromanfont[Mapping=tex-text]{Hoefler Text}
\setsansfont[Scale=MatchLowercase,Mapping=tex-text]{Gill Sans}
\setmonofont[Scale=MatchLowercase]{Andale Mono}

\title{\Huge \textbf{软件工程项目文档}  \\
\large ~~~~~~~ ———— “鲜活”社团管理平台
 ~\\ ~\\ ~\\ ~\\ ~\\ ~\\ ~\\ ~\\ ~\\ ~\\ ~\\} 
\author{
\large 1452822 洪嘉勇 \\
\large 1454093 夏陈   \\
\large 1452716 张尹嘉 \\
} 
%\date{}                                           % Activate to display a given date or no date

\begin{document}
\maketitle

% For many users, the previous commands will be enough.
% If you want to directly input Unicode, add an Input Menu or Keyboard to the menu bar 
% using the International Panel in System Preferences.
% Unicode must be typeset using a font containing the appropriate characters.
% Remove the comment signs below for examples.

% \newfontfamily{\A}{Geeza Pro}
% \newfontfamily{\H}[Scale=0.9]{Lucida Grande}
% \newfontfamily{\J}[Scale=0.85]{Osaka}

% Here are some multilingual Unicode fonts: this is Arabic text: {\A السلام عليكم}, this is Hebrew: {\H שלום}, 
% and here's some Japanese: {\J 今日は}.

\newpage
\tableofcontents
\newpage

\section{引言}
\subsection{背景}
近年来互联网、物联网等发展迅猛,高校校园在这波潮流中可以说是弄潮儿,很多互联网创新平台都在高校中运行得非常稳健。学校也正利用着这股潮流做着相应的改变,校园的管理系统正在渐渐地完善,比如图书馆的预览室已经可以在网上平台预约、校车也可以在网上抢票,但是跟学生们的课余生活最为密切的社团生活似乎并没有纳入学校网络化的考虑范围中。与此同时,我们的社团的活动也开始缺乏新意,社团的团长们缺乏好的点子,而在互联网+的时代下,去创建这样一二个信息共享的渠道是相对容易的,我们渴望在这个时代背景下为同济大学师生创建一个社团管理平台和社团活动信息共享平台,我们将其命名为“鲜活”。
\paragraph{同济大学社团发展现状} 
就目前同济大学校园而言,大多数关于社团的操作都非常地繁琐,通信的时间成本和经济成本都非常高,很多社团的发展因为制度问题而变得不健康。体现在以下方面:

\begin{enumerate}[1)]
\item 社团审批需要递交的材料多且审批等待时间长,将导致很多新兴社团痛失很多招生机会
\item 很多社团缺少展示的平台,很多社团因此默默无闻
\item 社团自身没有统一的管理规范,社团管理一团乱码
\item 社团通信流程复杂,通知时间成本高
\item 社团申请场地和申请海报张贴的流程繁琐,等待时间长
\end{enumerate}

\paragraph{线上平台的必要性}
现行的体制使得社团发展受到遏制,一个线上统一管理社团的平台已经成为师生共同的需求。而且传统的在食堂门口发传单等宣传方式早已受到师生们的反感,也为校园环保造成一定的压力,这些事情完全可以放在线上平台上来做。师生们渴望社团是一个又鲜又活的东西,而不是一个抽象的存在,我们需要想办法加强社团的存在感。而这一切都可以通过一个线上的社团管理平台实现。

\paragraph{线上平台的可行性}
在经过背景分析和技术论证之后,我们认为线上平台的可行性很强,具体为以下几点:
\begin{enumerate}[1)]
\item 互联网已经普及
\item 师生几乎人人都有手机
\item 同济大学“同心云”轻应用可以成为我们很好的发展平台
\end{enumerate}

\subsection{参考资料}

\subsection{假定和约束}

\subsection{用户的特点}

\section{功能需求}
\subsection{系统范围}
洪
\subsection{系统体系结构}
夏
\subsection{系统总体流程}
张
\subsection{需求分析}
夏
\subsection{功能建模}
夏
\subsection{数据建模}
洪
\subsection{行为建模}
张

\section{UI 需求}
洪

\section{非功能需求}
\subsection{性能需求}
洪
\subsubsection{精度}

\subsubsection{时间特性要求}

\subsubsection{输入输出要求}

\subsection{数据管理能力要求}
张
\subsection{安全及保密性要求}
张
\subsection{灵活性要求}
夏
\subsection{其他专门要求}
夏
\section{运行环境规定}
\subsection{设备}

\subsection{支持软件}
张
\section{需求跟踪}
夏
\section{附录}

\end{document}  
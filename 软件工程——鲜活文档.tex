% XeLaTeX can use any Mac OS X font. See the setromanfont command below.
% Input to XeLaTeX is full Unicode, so Unicode characters can be typed directly into the source.

% The next lines tell TeXShop to typeset with xelatex, and to open and save the source with Unicode encoding.

%!TEX TS-program = xelatex
%!TEX encoding = UTF-8 Unicode

\documentclass[UTF8]{ctexart}
\usepackage{geometry}                % See geometry.pdf to learn the layout options. There are lots.
\geometry{letterpaper}                   % ... or a4paper or a5paper or ... 
%\geometry{landscape}                % Activate for for rotated page geometry
%\usepackage[parfill]{parskip}    % Activate to begin paragraphs with an empty line rather than an indent
\usepackage{graphicx}
\usepackage{amssymb}
\usepackage{enumerate}

% Will Robertson's fontspec.sty can be used to simplify font choices.
% To experiment, open /Applications/Font Book to examine the fonts provided on Mac OS X,
% and change "Hoefler Text" to any of these choices.

\usepackage{fontspec,xltxtra,xunicode}
\defaultfontfeatures{Mapping=tex-text}
\setromanfont[Mapping=tex-text]{Hoefler Text}
\setsansfont[Scale=MatchLowercase,Mapping=tex-text]{Gill Sans}
\setmonofont[Scale=MatchLowercase]{Andale Mono}

\title{\Huge \textbf{软件工程项目文档}  \\
\large ~~~~~~~ ———— “鲜活”社团管理平台
 ~\\ ~\\ ~\\ ~\\ ~\\ ~\\ ~\\ ~\\ ~\\ ~\\ ~\\} 
\author{
\large 1452822 洪嘉勇 \\
\large 1454093 夏陈   \\
\large 1452716 张尹嘉 \\
} 
%\date{}                                           % Activate to display a given date or no date

\begin{document}
\maketitle

% For many users, the previous commands will be enough.
% If you want to directly input Unicode, add an Input Menu or Keyboard to the menu bar 
% using the International Panel in System Preferences.
% Unicode must be typeset using a font containing the appropriate characters.
% Remove the comment signs below for examples.

% \newfontfamily{\A}{Geeza Pro}
% \newfontfamily{\H}[Scale=0.9]{Lucida Grande}
% \newfontfamily{\J}[Scale=0.85]{Osaka}

% Here are some multilingual Unicode fonts: this is Arabic text: {\A السلام عليكم}, this is Hebrew: {\H שלום}, 
% and here's some Japanese: {\J 今日は}.

\newpage
\tableofcontents
\newpage

\section{引言}
洪
\subsection{背景}

\subsection{参考资料}

\subsection{假定和约束}

\subsection{用户的特点}

\section{功能需求}
\subsection{系统范围}
洪
\subsection{系统体系结构}
夏
\subsection{系统总体流程}
张
\subsection{需求分析}
夏
\subsection{功能建模}
夏
\subsection{数据建模}
洪
\subsection{行为建模}
张

\section{UI 需求}
洪

\section{非功能需求}
\subsection{性能需求}
洪
\subsubsection{精度}

\subsubsection{时间特性要求}

\subsubsection{输入输出要求}

\subsection{数据管理能力要求}
张
\subsection{安全及保密性要求}
张
\subsection{灵活性要求}
夏
\subsection{其他专门要求}
夏
\section{运行环境规定}
\subsection{设备}

\subsection{支持软件}
张
\section{需求跟踪}
夏
\section{附录}

\end{document}  